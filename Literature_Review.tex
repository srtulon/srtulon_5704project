
% ===========================================================================
% Title:
% ---------------------------------------------------------------------------
% to create Type I fonts type "dvips -P cmz -t letter <filename>"
% ===========================================================================
\documentclass[11pt]{article}       %--- LATEX 2e base
\usepackage{latexsym}               %--- LATEX 2e base
%---------------- Wide format -----------------------------------------------
\textwidth=6in \textheight=9in \oddsidemargin=0.25in
\evensidemargin=0.25in \topmargin=-0.5in
%--------------- Def., Theorem, Proof, etc. ---------------------------------
\newtheorem{definition}{Definition}
\newtheorem{theorem}{Theorem}
\newtheorem{lemma}{Lemma}
\newtheorem{corollary}{Corollary}
\newtheorem{property}{Property}
\newtheorem{observation}{Observation}
\newtheorem{fact}{Fact}
\newenvironment{proof}           {\noindent{\bf Proof.} }%
                                 {\null\hfill$\Box$\par\medskip}
%--------------- Algorithm --------------------------------------------------
\newtheorem{algX}{Algorithm}
\newenvironment{algorithm}       {\begin{algX}\begin{em}}%
                                 {\par\noindent --- End of Algorithm ---
                                 \end{em}\end{algX}}
\newcommand{\step}[2]            {\begin{list}{}
                                  {  \setlength{\topsep}{0cm}
                                     \setlength{\partopsep}{0cm}
                                     \setlength{\leftmargin}{0.8cm}
                                     \setlength{\labelwidth}{0.7cm}
                                     \setlength{\labelsep}{0.1cm}    }
                                  \item[#1]#2    \end{list}}
                                 % usage: \begin{algorithm} \label{xyz}
                                 %        ... \step{(1)}{...} ...
                                 %        \end{algorithm}
%--------------- Figures ----------------------------------------------------
\usepackage{graphicx}

\newcommand{\includeFig}[3]      {\begin{figure}[htb] \begin{center}
                                 \includegraphics
                                 [width=4in,keepaspectratio] %comment this line to disable scaling
                                 {#2}\caption{\label{#1}#3} \end{center} \end{figure}}
                                 % usage: \includeFig{label}{file}{caption}
\usepackage{biblatex}
\addbibresource{my-bibliography.bib}
% ===========================================================================
\begin{document}
% ===========================================================================

% ############################################################################
% Title
% ############################################################################

\title{An Effective way to optimize performance and battery life of smart devices using multi-kernel approach}


% ############################################################################
% Author(s) (no blank lines !)
\author{
% ############################################################################
Sadid Rafsun Tulon\\
Faculty of Engineering\\
University of Ottawa\\
Ottawa, Canada K1N 6N5\\
{\em stulo080@uottawa.ca}
% ############################################################################
} % end-authors
% ############################################################################

\maketitle

% ############################################################################
% Abstract
% ############################################################################
\begin{abstract}
A very concise summary of the problem addressed and solution presented in this paper.
\end{abstract}


% ############################################################################
\section{Introduction} \label{intro}
% ############################################################################

Introduce your project topic (start from parallel computing in
general and lead to your particular topic). Describe your project
goals. Describe what you have achieved in your project. Outline
the structure of your paper: In Section~\ref{litrev}, we will
review the relevant literature.  ...
Section~\ref{concl} concludes the paper.


% ############################################################################
\section{Literature Review} \label{litrev}
% ############################################################################

Give an overview of the relevant literature. Cite all relevant
papers, like \cite{DEL07}, \cite{PD07}, \cite{DER07}, \cite{LDR07},
\cite{DLX06}, \cite{CDE06}, and \cite{DFL06}. Outline for each paper
the relevant results in relation to your project. Make sure that you
don't just list all relevant papers in random order. Devise a scheme
to group papers by subject. The goal of this section is to present
to the reader the state-of-the-art in the field selected for your
project.

% % Bibliography
% ############################################################################

\printbibliography     %loads my-bibliography.bib

% ============================================================================
\end{document}
% ============================================================================
