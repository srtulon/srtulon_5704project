% ===========================================================================
% Title:
% ---------------------------------------------------------------------------
% to create Type I fonts type "dvips -P cmz -t letter <filename>"
% ===========================================================================
\documentclass[11pt]{article}       %--- LATEX 2e base
\usepackage{latexsym}               %--- LATEX 2e base
%---------------- Wide format -----------------------------------------------
\textwidth=6in \textheight=9in \oddsidemargin=0.25in
\evensidemargin=0.25in \topmargin=-0.5in
%--------------- Def., Theorem, Proof, etc. ---------------------------------
\newtheorem{definition}{Definition}
\newtheorem{theorem}{Theorem}
\newtheorem{lemma}{Lemma}
\newtheorem{property}{Property}
\newtheorem{observation}{Observation}
\newtheorem{fact}{Fact}
\newenvironment{proof}           {\noindent{\bf Proof.} }%
                                 {\null\hfill$\Box$\par\medskip}
%--------------- Algorithm --------------------------------------------------
\newtheorem{algX}{Algorithm}
\newenvironment{algorithm}       {\begin{algX}\begin{em}}%
                                 {\par\noindent --- End of Algorithm ---
                                 \end{em}\end{algX}}
\newcommand{\step}[2]            {\begin{list}{}
                                  {  \setlength{\topsep}{0cm}
                                     \setlength{\partopsep}{0cm}
                                     \setlength{\leftmargin}{0.8cm}
                                     \setlength{\labelwidth}{0.7cm}
                                     \setlength{\labelsep}{0.1cm}    }
                                  \item[#1]#2    \end{list}}
                                 % usage: \begin{algorithm} \label{xyz}
                                 %        ... \step{(1)}{...} ...
                                 %        \end{algorithm}
%--------------- Figures ----------------------------------------------------
\usepackage{graphicx}
\usepackage{float}
\usepackage{graphicx}
\usepackage{caption}
\usepackage{subcaption}

\newcommand{\includeFig}[3]      {\begin{figure}[htb] \begin{center}
                                 \includegraphics
                                 [width=4in,keepaspectratio] %comment this line to disable scaling
                                 {#2}\caption{\label{#1}#3} \end{center} \end{figure}}
                                 % usage: \includeFig{label}{file}{caption}


% ===========================================================================
\begin{document}
% ===========================================================================

% ############################################################################
% Title
% ############################################################################

\title{An Effective way to optimize performance and battery life of smart devices using multi-kernel approach}


% ############################################################################
% Author(s) (no blank lines !)
\author{
% ############################################################################
Sadid Rafsun Tulon\\
School of Computer Science\\
University of Ottawa\\
Ottawa, Canada\\
{\em stulo080@uottawa.ca}
% ############################################################################
} % end-authors
% ############################################################################

\maketitle

% ############################################################################
% Abstract
% ############################################################################
\begin{abstract}
Improving the performance and battery life of smart devices is modern technology's most vital goal. As lowing transistor size becomes increasingly complex and battery technology has plateaued, other methods to increase performance and decrease power consumption are being explored relentlessly. While there are many techniques available, this paper addresses how to improve the performance and battery life of modern battery-powered smart devices using a new multi-kernel system technique. The main goal of this paper is to examine the potential of multi-kernel systems and compare them against Linux systems.
\end{abstract}

% ############################################################################
\section{Introduction} \label{intro}
% ############################################################################

Smart devices have become an integral part of our life. Many of the smart devices are battery-powered. The current battery technology is the most significant handicap of portable smart devices right now. As CPUs and systems on a chip(SoC) improve their efficiency, things are getting better. However, modern software also demands more processing power, which faster drains the battery. The lightweight kernels can extend battery life much further, but their software compatibility is far from ideal. Now that battery technology has reached a plateau and transistors in the semiconductors are also close to reaching their size limit, we are in need of an alternate way of achieving better battery life without compromising performance and compatibility. A new approach using multiple operating system (OS) kernels can help mitigate this problem. Almost all modern chips are multi-core. We can use these multi-core chips to run multiple OS kernels at the same time. Ideally, one of the kernels is a lightweight kernel for better performance and battery life and a monolithic kernel for better compatibility. The system can switch between the kernels, run them simultaneously, or stop them from running depending on the need. This way, we will be able to get better battery life while still retaining the software compatibility. This research is an effort to optimize performance, compatibility, and battery life by utilizing a multi-kernel approach while using existing battery technology. 

In this project, the idea is to research if we fit two operating systems for both cores of multi-core processors individually and how they perform.

% ############################################################################
\section{Literature Review} \label{litrev}
% ############################################################################

\subsection{Different types of kernels}
The current hardware technology is changing rapidly. The software is struggling to keep up with this rapid change. Thus it has become very challenging to optimize software for a specific hardware\cite{DEL02}. So we need a new OS structure that can be as scalable as needed. This is where LWKs and unikernels come in handy. Multi-core processors are used in the majority of current systems. We can leverage this to our advantage.

Light Weight Kernels (LWKs) are not a new concept. They have been here for at least 30 years \cite{DEL04}. However, they are only gaining popularity in these past couple of years because of High-Performance Computing (HPC). HPC necessitates tremendous scalability, which the Full-Weight Kernel (FWKs) take a long time to adjust to. LWKs work exceptionally well with highly scalable systems as they have a simple codebase.Table \ref{table:1} describes the Design goals of LWKs and FWKs\\


\begin{table}
\begin{tabular}{ c c c }
\hline
 Design goal & LWK & FWK \\ [0.5ex] 
\hline
 Target & massively parallel systems & laptops, desktops, servers \\ 
 Support & scalable applications & everything under the sun \\  
 Dev. environment for & parallel applications & business, games, commerce, etc. \\
 Emphasis & effciency & functionality  \\ 
 Resources & maximize use & fair sharing, QoS  \\
 Time to completion & minimal & when needed  \\
\end{tabular}
\caption{Design goals of LWKs and FWKs \cite{DEL04}}
\label{table:1}
\end{table}

There is also a new type of single address space operating system called unikernels. They are minimalistic single-purpose operating systems made from library os [unikernel]. As they are very specialized, it is easy to optimize them for hardware. Like LWKs, unikernels are also very lightweight, sometimes unikernels are lighter than LWKs [azealea..........................]. 

Figure~\ref{fig1} shows application stack on monolithic OS vs Unikernel.

\includeFig{fig1}{Figures/figure.png}{Application stack on monolithic OS vs Unikernel \cite{DEL11}}



In this paper, we will be treating both unikernal and LWK as the sub kernel as they both have better performance and efficiency than FWKs and have limited compatibility with the standard  APIs \cite{DEL03}, \cite{DEL01}. Another major issue of LWKs and unikernels are driver support. As they have a simple codebase mainly focusing on performance, they are missing many advanced features that FWKs have \cite{DEL03} [unikernel].


\subsection{Multi-Kernel systems}
Multi-kernel operating systems have been presented as a way to improve system performance and efficiency without sacrificing software and hardware compatibility. Multikernel OS runs both FWK and sub kernel side by side divides their tasks \cite{DEL01}. The sub kernel is usually responsible for the high-performance tasks, and FWK is there for ensuring compatibility, driver support and serves as a backup system \cite{DEL03}. In high-performance supercomputers, this system performs admirably. 

Figure~\ref{fig2} shows the system architecture of a multikernel system.

\includeFig{fig2}{Figures/figure2.png}{System Architectural Overview of Multi-kernel \cite{DEL04}}

This paper will explore if we can use this multi-kernel method on battery-powered smart devices. We will compare performance and power efficiency between a single Linux kernel system and a multi-kernel system. As batteries have a very small capacity, ensuring longer battery life while maintaining consistent performance is crucial for this project. Thus performance and power efficiency will be the main focuses of this research project.

% ############################################################################
\section{Problem Statement} \label{problemStatement}
% ############################################################################

\begin{enumerate}
\item A concise statement of the question that your paper tackles.
\item Justification, by direct reference to your Literature Review, that your question is previously unanswered.
\item Discussion of why it is worthwhile to answer this question.
\end{enumerate}

% ############################################################################
\section{Proposed Solution: ...} \label{proposedSolution}
% ############################################################################

This part of the paper is much more free-form. It may have one or several sections and subsections. But it all has only one purpose: to convince the reader that you answered the question or solved the problem that you set for yourself. In this section you will for example present new algorithms you developed and your implementation of these new algorithms.

 Figure~\ref{fig3} is an example
of a drawing created with {\em mdraw} or {\em epsfig}.

\includeFig{fig3}{Figures/figure-2}{XYZ and Hilbert Packings}
% usage: \includeFig{label}{file}{caption}

% ############################################################################
\section{Experimental Evaluation} \label{expEval}
% ############################################################################

The following experiments are done with a machine using Intel core I5 4500H 4 core CPU  clocked at 2.50 GHz, 8 gigabytes of ram. Primary operating systems were Ubuntu and CentOs depending on the situation. We tried to compile and test various open-source LWKs and Unikernels such as Azalea, HermitCore, mckernel, MirageOs, Unikraft. 
Table \ref{table:2}  describes the summery of each experiment.



From Table \ref{table:2}, we can see that only Unikraft fully worked on our system. The rest of the kernels refused to compile or run. 

We also tried to compile and test the kernels mentioned above on the Raspberry pi 4 4 gigabyte model. Only helloworld module of Unikraft ran properly.  

Figure~\ref{fig4} is the bootscreen of Unikraft running Flask server.

\includeFig{fig4}{Figures/unikraftboot.png}{Bootscreen of Unikraft running Flask server}






We suspect such a situation occurred because these kernels are built for server CPUs such as Intel Xeon, AMD Epyc, and Fujitsu A64FX. Almost all of the LWKs and unikernels and specialized for servers, so it is understandable that consumer-grade CPUs are incompatible with them. 


Figure~\ref{fig5} is comparison of power consumption between Unikraft Flask server and Linux native Flask server without including whole Linux power consumption. 

\includeFig{fig5}{Figures/wolinux.jpg}{Unikraft Flask server vs Linux native Flask server (without OS) power consumption}

Figure~\ref{fig6} is comparison of power consumption between Unikraft Flask server and Linux native Flask server without including whole Linux power consumption.

\includeFig{fig6}{Figures/wlinux.jpg}{Unikraft Flask server vs Linux native Flask server (with OS) power consumption}

	




\clearpage

% ############################################################################
\section{Conclusions} \label{concl}
% ############################################################################

You generally cover three things in the Conclusions section.
\begin{enumerate}
\item Conclusions
\item Summary of Contributions
\item Future Research
\end{enumerate}

Conclusions are not a rambling summary of the thesis: they are short, concise statements of the inferences that you have made because of your work. All conclusions should be directly related to the research question.

The Summary of Contributions will be much sought and carefully read by the readers. Here you list the contributions of new knowledge that your paper makes. Of course, the paper itself must substantiate any claims made here. There is often some overlap with the Conclusions, but that’s okay.

The Future Research should indicate interesting new problems arising from your work. No paper ever solves everything. In fact, the best research papers lead to new research questions for other researchers to work on.


% ############################################################################
% Bibliography
% ############################################################################
\bibliographystyle{plain}
\bibliography{my-bibliography}     %loads my-bibliography.bib

% ============================================================================
\end{document}
% ============================================================================
